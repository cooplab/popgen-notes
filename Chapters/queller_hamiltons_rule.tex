\paragraph{Breeding values}

\gc{Here be dragons, this section has not be worked through}
Before we turn to our discussion of the response to selection it will
help to define one more quantity, the breeding value. 
The breeding value of an individual is the mean phenotype of the
offspring of this individual averaged over all possible matings in the
population. This is $X_{Ai}$.

For an individual $i$ we can write the the breeding value of an individual $G_X$ as
\begin{equation}
G_i = \alpha_G + \beta_{GP} X_i   +\epsilon_G %equation 10.4 Falconer
\end{equation}
where $\beta_{GX}  Cov(X,G)/Var(X) =h^2$ 


The breeder's equation is in fact a statement of the change in the mean
breeding value of the population. To see this consider the fact that
we can write the change in mean breeding value within a generation due
to selection as 
\begin{equation}
R=\E[G_i] = \beta_{GP} \bar{X_{}}-\bar{P_X}
\end{equation}

We can also write the breeder's equation as 
\begin{equation} 
R =\delta \bar{G} = Cov(G,W)
\end{equation}
i.e. the mean change in phenotypes over the generations is due to the
fact that our breeding values covaries with fitness 
this is the Price equation. 

If  $\epsilon_G$ and G are uncorrelated then
\begin{equation}
\delta \bar{G} = Cov(G,W) = \beta_{GP} Cov(P,W) = h^2 S
\end{equation}
we can also write
\begin{equation}
W = \alpha_W + \beta_{WP} P +\epsilon_W
\end{equation}
where $\beta_{WP} = Cov(W,P)/Var(P)$. If $\epsilon_W$ and $G$ are uncorrelated

\begin{equation}
\delta \bar{G} = Cov(G,W) = \beta_{WP} Cov(G,P) 
\end{equation}


\subsubsection{Hamiliton's Rule and the evolution of altruistic and
  selfish behaviours}

\gc{Here be dragons, this section has not be worked through}
We can use our quantitative genetics framework to gain some simple
intuition for when altruistic behaviours should evolve through kin selection. To do this we
can follow Queller's (1992) treatment, to rederive Hamiliton's rule in
a quantitative genetics framework (Hamilition original work did this in a
population genetics framework).

So lets imagine that individuals interact in pairs, with our focal
individual $i$ being paired with an individual $j$.  
Imagine that individuals have two possible phenotypes $X=1$ or $0$,
corresponding to providing or withholding some small act of `Altruism'
(we could just as easily flip these labels and call them a unselfish
act and a selfish act respectively). Providing the altruistic act has a cost $C$ to the fitness of our
individual, with failing to provide this act has no cost. Receiving this
altruistic act confers a fitness benefit $B$ (not receiving it has no benefit).
Hamiliton's Rule is that such a trait will spread through the
population if the average coefficient of relatedness between two
individuals is $r$, i.e. the probability that a gene picked randomly 
from each at the same locus is identical by descent, such that $C<rB$.
The intuition here is that while our focal individual $i$ is losing
out of some reproductive output, the alleles underlying the behavior
are still spreading as this cost is outweighed by the transmission of these 
alleles through a related individual (note that this means that the
allele is not acting in an self sacrificing manner, even though
individuals may as a result). 
  

We will assume that our focal $i$ individual's relative fitness can be written as 
\begin{equation}
W(X_i,X_j)= W_0 + W_i +W_j
\end{equation}
where $W_i$ is the contribution of the fitness of the individual $i$ due
to this phenotype, and $W_j$ is the contribution to our
individual $i$'s fitness due to the $j$'s behaviour (i.e. phenotype).
With the benefit $B$ and cost $C$ our $W(X_i,X_j)$ are depicted in
Figure XXX. We can write our fitness component for our individual's fitness
as 
\begin{equation}
W_i = \alpha_i + \beta_{W_i P_i} X_i + \epsilon_{W_i}
\end{equation}
and
\begin{equation}
W_j = \alpha_j + \beta_{W_j P_j} P_j + \epsilon_{W_j}
\end{equation}
our slopes $\beta_{W_i P_i} = Cov(W_i,P_i)/Var(P_i)$ and $\beta_{W_j
  P_j} = Cov(W_j,P_j)/Var(P_j)$ i.e. the regression our our individual
$i$s fitness components on his and his neighbour's phenotype. From 
Figure XXX these slopes are just our cost and benefit.

Using the price equation, assuming that the residuals are not
correlated with our genotype we arrive at
\begin{equation}
\delta \bar{G}_X = \beta_{W_i P_i} Cov(G_i,P_i) +  \beta_{W_j P_j} Cov(G_i,P_j) 
\end{equation}
so our  altruistic trait will increase in frequency in the
population only if $\delta \bar{G}_X >0$, i.e. that 
\begin{equation}
0<\beta_{W_i P_i}  + \beta_{W_j P_j} \frac{Cov(G_i,P_j)}{Cov(G_i,P_i)} 
\end{equation}
substituting in our slopes we arrive at 
\begin{equation}
C<  B \frac{Cov(G_i,P_j)}{Cov(G_i,P_i)} 
\end{equation}
i.e. our altruistic trait will increase if the cost is outweighed by
the benefit weighted by a factor $\frac{Cov(G_i,P_j)}{Cov(G_i,P_i)}$. The term $\frac{Cov(G_i,P_j)}{Cov(G_i,P_i)}$ 
is the ratio of the ratio of covariance between our focal
individual's genotype and their phenotype and that of their
interacting neighbour. 