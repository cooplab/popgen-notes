\section{Genetic drift and Neutral alleles}






\subsection{Comparing polymorphism and divergence}


\subsection{Deviations from the constant population model.}
We've seen previously that changes in our population size can be
captured by an effective population size. However, this will only be a
useful measure if population sizes vary rapidly enough, that the
harmonic mean effective population size over short time periods ($\ll
N_e$ generations) is representative of the effective population size averaged over
longer time periods. If this is not the case there is no one effective
population size, as we can not approximate our rate of drift by a
single constant population. Furthermore, we've ignored the effect of
population structure and selection which will violate our modeling
assumptions. \\

We can hope to detect violations from our constant population size
neutral model, by comparing aspects of our dataset to their expectations
and distributions under our neutral model. \\

For example we have devised two estimates of $\theta$,
$\widehat{\theta_{\pi}}$ and $\widehat{\theta_{W}}$, using
expectations of different aspects of our data (pairwise diversity and
number of segregating sites respectively). Under our constant neutral
model if we have sufficient data those two estimates should be
equal to each other on average. But if there's some violation of our model they might not
be. So one test statistic might be to take
\begin{equation}
D = \widehat{\theta_{\pi}} - \widehat{\theta_{W}}
\end{equation}
which will be zero in expectation if our data was generated by a
neutral constant population model.




\newpage
